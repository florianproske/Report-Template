%!TEX root = main.tex
% !TeX program = xelatex
%% SETUP OF CUSTOM DOCUMENT-PROPERTIES

%% refer to the readme for further information about the content of this file


% choose which type of report you want to write
\usepackage[
    Internship,			% Internship template
    % Study,				% Study template
    % Bachelor			% Bachelor template
    TitlepageCenter,			% Titlepage centered (especially for long titles)
    DbSystelTheme,			% DB Systel Theme (with DB Systel Claim and special colors)
    %DefaultTheme			% DB Theme
]{optional}


 

% uncomment if you need a Confidentiality Clause
%\newcommand{\activateconfidentialityclause}{def}

% uncomment if you need a second line for the departmentname
%\newcommand{\activatelongdepartment}{def}

% choose your language by uncommenting the variant you like:
    % english with german parts
    %\setdefaultlanguage{english}
    %\setotherlanguage[variant=german, latesthyphen=true]{german}
%  or
    % fully german
    \setdefaultlanguage[variant=german, latesthyphen=true]{german}


% place your information into the following fields:
\newcommand{\studentname}{Max Mustermann}
\newcommand{\matrikelno}{123456}
\newcommand{\kurs}{MA-TINF...}

\newcommand{\companyname}{DB Systel GmbH}
\newcommand{\companylocation}{Frankfurt am Main}

\newcommand{\studiengang}{Angewandte Informatik}
\newcommand{\dhbwDE}{Dualen Hochschule Baden-Württemberg Mannheim}
\newcommand{\dhbwEN}{Baden-Wuerttemberg Cooperative State University Mannheim}

\newcommand{\reporttitle}{Wie erstelle ich einen sch{\"o}nen Praxisbericht}

\newcommand{\timerange}{... Wochen}
\newcommand{\handoverdate}{01.01.1970}
\newcommand{\city}{Foobar}


% place your information into the block which corresponds to the type you have choosen at the beginning
\opt{Internship}{
    \newcommand{\reportmodule}{Tnn000}
    \newcommand{\reportstart}{01.01.1970}
    \newcommand{\reportend}{01.01.1970}
	\newcommand{\department}{F.I ...}
	\newcommand{\longdepartment}{...}
	\newcommand{\companymanager}{...}
}
\opt{Study}{
    \newcommand{\reportsemester}{n th}
    \newcommand{\prof}{Foobar}
}
\opt{Bachelor}{
    \newcommand{\academicdegree}{Bachelor of Science}
    \newcommand{\companymanager}{Foobar}
    \newcommand{\prof}{Foobar}
}

% PrimaryAccentColor DB Systel Gelbgruen or Deutsche Bahn Rot
\opt{DbSystelTheme}{\definecolor{PrimaryAccentColor}  	{RGB}{120, 190, 20}}
\opt{DefaultTheme}{\definecolor{PrimaryAccentColor}  	{RGB}{240, 20, 20}}

% Deutsche Bahn Dunkelgrau & Grau
\definecolor{SecondaryAccentColor}	{RGB}{100, 105, 115}
\definecolor{TertiaryAccentColor} 	{RGB}{135, 140, 150}

% Deutsche Bahn primary colors
\definecolor{DbRot}     		{RGB}{240, 20, 20}
\definecolor{DbBlau}     		{RGB}{10, 30, 110}
\definecolor{DbDunkelgrau}	{RGB}{100, 105, 115}
\definecolor{DbGrau}     		{RGB}{135, 140, 150}
\definecolor{DbHellgrau}     	{RGB}{200, 205, 210}
\definecolor{DbWeissgrau}     	{RGB}{225, 230, 235}

% Deutsche Bahn secondary colors
\definecolor{DbGelb}     		{RGB}{255, 205, 0}
\definecolor{DbSonnengelb}     	{RGB}{255, 175, 0}
\definecolor{DbVerkehrsorange}	{RGB}{255, 140, 0}
\definecolor{DbGelbgruen}     	{RGB}{120, 190, 20}
\definecolor{DbGrasgruen}     	{RGB}{0, 170, 0}
\definecolor{DbSignalblau}     	{RGB}{0, 100, 200}
\definecolor{DbCurrygelb}     	{RGB}{160, 130, 0}
\definecolor{DbOrangebraun}   	{RGB}{175, 90, 0}
\definecolor{DbRubinrot}     	{RGB}{165, 5, 45}
\definecolor{DbPurpurviolett}  	{RGB}{95, 20, 120}
\definecolor{DbVerkehrsblau}  	{RGB}{0, 100, 140}
\definecolor{DbOpalgruen}     	{RGB}{0, 110, 115}
\definecolor{DbTuerkisgruen}  	{RGB}{0, 120, 65}
\definecolor{DbFarngruen}     	{RGB}{90, 110, 0}

% if you want to, you can also change the value of the supporting colors:
\definecolor{SupportingGreen}     {rgb}{0.00, 0.55, 0.14}
\definecolor{SupportingOrange}    {rgb}{0.97, 0.66, 0.15}
\definecolor{SupportingDeepOrange}{rgb}{0.93, 0.30, 0.18}
\definecolor{SupportingYellow}    {rgb}{0.99, 0.98, 0.12}
\definecolor{SupportingCherry}    {rgb}{0.80, 0.00, 0.34}
\definecolor{SupportingPurple}    {rgb}{0.43, 0.17, 0.55}
